\chapter{Execution}
\label{execution}

The execution of a CX program must start with the creation of a CX
call element. This call must be associated with the main function,
from the main module (as explained in Section \ref{overview}, every CX
program must contain a main function and main module, which act as an
entry point). After creating this call, it must be appended to the
ordered collection of calls, and start the execution of the program
with this initial call.

Execution can be stepped in CX.\@ This implies that the process must be
aware of the amount of calls that have been added to the call stack
since the last time the program was paused. The execution process must
also have access to the call stack, and have permission to manipulate
the call stack: if a program is required to go back \textit{N} steps,
\textit{N} calls must be popped from the call stack. If a program goes
back \textit{N} steps, the information from these \textit{N} calls
must be discarded, i.e., any change performed by the \textit{N} calls
must be rolled back, and any local definitions contained in the state
of the calls must be discarded.

Every time a CX program is instructed to evaluate a function, a call
is created and pushed onto the call stack. A call must have a
reference to the function which is going to be evaluated, the line
number at which the program execution is at (i.e., the index of the
expression in a CX function definition). A reference to the return
address (the call which created the current call) must be accessible;
once a call finishes its execution, it can return an output value by
appending a CX definition to the state of the caller by using the
this reference, and it must be popped out from the stack. A call
finishes its execution once all the expressions contained in
the function being called have been evaluated. Whenever a call is
created, an internal counter is incremented so program stepping can be
performed.

If a function did not explicitly assign values to its output
parameters, a call should return the output values returned by its
last expression.

The execution process is also in charge of ensuring that all the
arguments sent to every function call are of the correct type, and that
the correct number of arguments have been sent. If there is a mismatch
either in the arguments types or in the arguments number, the
execution process should halt the program without killing the
program's process (see Section \ref{call-stack-debugging} below).

Every expression contained in the function definition of the callee
must be evaluated by the execution process. Before performing the
evaluation of an expression, its arguments also need to be
evaluated. If an argument is another function evaluation, another call
will be created. If the argument is an identifier which is referencing
a value determined in a previous expression, a global definition, or a
function input or output parameter, the identifier needs to be
resolved in order to find the actual value and type that the
identifier is referring to. If the argument is a native function (see
Section \ref{native-function-calling}) (determined by checking against
a list of all the native functions defined in the core module), the
execution process sends the CX argument elements to the required
function. Native functions are defined in the host language, and
they provide basic functionality, like arithmetic operations. The
arguments sent to a native function need to be converted to be readable
by the host language. This conversion process usually involves using
the argument's type to know how to interpret the bytes stored in the
argument's value. Once the native function has finished processing its
arguments, the output needs to be converted back to a CX argument
element. Lastly, if an argument to a function call is a literal (e.g.,
10, ``a string'', 3.14), these are directly sent to the callee.

If the program is seen as an abstract syntax tree, every leaf in a CX
AST will always be a native function, i.e., every CX program could be
decomposed into a structured series of native calls.

\section{Call Stack Debugging}
\label{call-stack-debugging}

A CX program should be resilient to crashing. If possible, a CX
program should always halt its execution when an error is encountered,
inform the programmer about the error, and give the programmer the
opportunity to change the program's state and structure.

Whenever an error is encountered, CX should enter REPL mode (see
Section \ref{cx-repl}), if not already in it, so the programmer has
access to the meta-programming commands (see Section
\ref{meta-programming}) to modify the program's structure and
execution. The last call in the call stack will be discarded or popped
out (which is equivalent to stepping back one step), and the
programmer can start altering the program's structure. When an error
is encountered, a CX implementation should give the program as much
relevant information as possible. At a minimum, when an error is
encountered, the function which caused the error, the arguments that
were sent to the function, the line number of such expression, and a
call stack trace should be printed for the programmer. The amount of
calls in the call stack to be printed is not specified.

\section{Native Function Calling}
\label{native-function-calling}

When an expression is called, the execution process must look for the
name of the callee's operator in the collection of functions attached
to the current module, or the collection of functions of imported
modules if the identifier name starts with the name of a module
followed by a period (see Section \ref{identifiers} for more
information about identifiers). If the name is found in one of these
modules, the actual CX function element is retrieved and executed with
the arguments provided in the caller's function definition. If the
name is not found in the current module or any of the imported
modules, the name is looked for in the collection of native functions
in the core module. If the name is not found in the collection of native
functions, an error should be raised, and if it is found, the function
is called with the provided arguments.

Native functions are implemented in the programming language in which
the CX implementation is being programmed in. A native function should
not call any CX functions, and this implies that a native function
can be considered as a terminal node in the CX program
structure. All native functions must be able to receive CX argument
elements, and must be able to convert them to
native arguments to the host programming language. For example, the
native function ``addI32'' should receive two CX argument elements
which hold 32 bit integers as values, and are of type ``i32.'' The
native function can have the responsibility of checking if the
arguments are of the expected types, and then should proceed with the
conversion from CX 32 bit integers, to the host programming language's
equivalents. The integers should be added together and the resulting
value must be converted to a CX 32 bit integer, which must be attached
to a CX argument element that will serve as the native function's
output parameter's value.

\section{Compilation}
\label{compilation}

A compilation process can be called at any time (see Section
\ref{compilation}) by a CX program. Compilation is in charge of
optimizing the program structure to require fewer system resources, such as
processing power and memory. A CX program element must have access to
a byte array which represents a memory heap. The compilation process
can place certain values from different elements in the heap to access
them in a constant time.
