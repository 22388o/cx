% Lexical structure

%%\vfill\eject
\chapter{Lexical conventions}

\section{Identifiers}
\label{syntaxsection}

\begin{scheme}
!\ \$ \% \verb"&" * + - . / :\ < = > ? @ \verb"^" \verb"_" \verb"~" %
\end{scheme}

\begin{scheme}
...                      {+}
+soup+                   <=?
->string                 a34kTMNs
lambda                   list->vector
q                        V17a
|two words|              |two\backwhack{}x20;words|
the-word-recursion-has-many-meanings%
\end{scheme}

\begin{entry}{%
{\cf{}\#!fold-case}\sharpbangindex{fold-case}\\
{\cf{}\#!no-fold-case}\sharpbangindex{no-fold-case}}

These directives can appear anywhere comments are permitted (see
section~\ref{wscommentsection}) but must be followed by a delimiter.
They are treated as comments, except that they affect the reading
of subsequent data from the same port. The {\cf{}\#!fold-case} directive causes
subsequent identifiers and character names to be case-folded
as if by {\cf string-foldcase} (see section~\ref{stringsection}).
It has no effect on character
literals.  The {\cf{}\#!no-fold-case} directive
causes a return to the default, non-folding behavior.
\end{entry}



\section{Whitespace and comments}
\label{wscommentsection}

\begin{scheme}
\#|
   The FACT procedure computes the factorial
   of a non-negative integer.
|\#
(define fact
  (lambda (n)
    (if (= n 0)
        \#;(= n 1)
        1        ;Base case: return 1
        (* n (fact (- n 1))))))%
\end{scheme}


\section{Other notations}

\section{Datum labels}\unsection
\label{labelsection}

\begin{scheme}
(let ((x (list 'a 'b 'c)))
  (set-cdr! (cddr x) x)
  x)                       \ev \#0=(a b c . \#0\#)
\end{scheme}

\begin{scheme}
\#1=(begin (display \#\backwhack{}x) \#1\#)
                       \ev \scherror%
\end{scheme}

