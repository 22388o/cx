\chapter{Implementation of a CX Language}
\label{implementation-of-a-cx-language}

This Section describes the most important aspects of the CX
implementation that can be found in
\url{http://github.com/SkycoinProject/cx}, in order to clarify the concepts
covered in the Sections regarding the CX specification. The code
presented in the code listings in this Section is not guaranteed to be
up-to-date as the code present in the official repository.

\section{Adders}
\label{implementation-adders}



\begin{lstlisting}[float=*, caption=Example of an AddModule adder, label=implementation-addmodule]
func (cxt *CXProgram) AddModule (mod *CXModule) *CXProgram {
	mod.Context = cxt
	cxt.CurrentModule = mod
	found := false
	for i, md := range cxt.Modules {
		if md.Name == mod.Name {
			cxt.Modules[i] = mod
			found = true
			break
		}
	}
	if !found {
		cxt.Modules = append(cxt.Modules, mod)
	}
	return cxt
}
\end{lstlisting}

\begin{lstlisting}[float=*, caption=Example of an AddObject adder, label=implementation-addobject]
func (mod *CXModule) AddObject (obj string) *CXModule {
	mod.Objects = append(mod.Objects, obj)

	return mod
}
\end{lstlisting}

\begin{lstlisting}[float=*, caption=Example of an AddClauses adder, label=implementation-addclauses]
func (mod *CXModule) AddClauses (clauses string) *CXModule {
	mod.Clauses = clauses
	return mod
}
\end{lstlisting}

\section{Affordances}
\label{implementation-affordances}

\begin{lstlisting}[float=*, caption=]
func (strct *CXStruct) GetAffordances() []*CXAffordance {
	affs := make([]*CXAffordance, 0)
	mod := strct.Module
	types := make([]string, len(BASIC_TYPES))
	copy(types, BASIC_TYPES)
	for _, s := range mod.Structs {
		types = append(types, s.Name)
	}
	for _, imp := range mod.Imports {
		for _, strct := range imp.Structs {
			types = append(types, concat(imp.Name, ".", strct.Name))
	       	}
	}
	for _, typ := range types {
		fldGensym := MakeGenSym("fld")
		fldType := MakeType(typ)
		affs = append(affs, &CXAffordance{
			Description: concat("AddField ", fldGensym, " ", typ),
			Action: func() {
				strct.AddField(MakeField(fldGensym, fldType))
			}})
	}
	return affs
}
\end{lstlisting}

\section{Execution}
\label{implementation-execution}

\begin{lstlisting}[float=*, caption=]
func (cxt *CXProgram) Run (withDebug bool, nCalls int) {
	var callCounter int = 0
	if cxt.CallStack != nil && len(cxt.CallStack.Calls) > 0 {
		lastCall := cxt.CallStack.Calls[len(cxt.CallStack.Calls) - 1]
		lastCall.call(withDebug, nCalls, callCounter)
	} else {
		if mod, err := cxt.SelectModule("main"); err == nil {
			if fn, err := mod.SelectFunction("main"); err == nil {
				state := make([]*CXDefinition, 0)
				mainCall := MakeCall(fn, state, nil, mod, mod.Context)
				cxt.CallStack.Calls = append(cxt.CallStack.Calls, mainCall)
				mainCall.call(withDebug, nCalls, callCounter)
			}
		} else {
			fmt.Println(err)
		}
	}
}
\end{lstlisting}

\begin{lstlisting}[float=*, caption=]
if nCalls > 0 && callCounter >= nCalls {
		return
}
callCounter++
\end{lstlisting}

\begin{lstlisting}[float=*, caption=]
if call.Line >= len(call.Operator.Expressions)
\end{lstlisting}

\begin{lstlisting}[float=*, caption=Popping out an Element from the Stack]
call.Context.CallStack.Calls = 
call.Context.CallStack.Calls[:len(call.Context.CallStack.Calls) - 1]
\end{lstlisting}

\begin{lstlisting}[float=*, caption=]
panic(fmt.Sprintf("output '\%s' is of type '\%s'; '\%s' requires output type '\%s'",
out.Name, out.Typ.Name, call.Operator.Name, call.Operator.Outputs[i].Typ.Name))
\end{lstlisting}

\begin{lstlisting}[float=*, caption=]
\end{lstlisting}

\section{Native Functions}
\label{implementation-native-functions}

\begin{lstlisting}[float=*, caption=Example of a Native Function]
func divI32 (arg1 *CXArgument, arg2 *CXArgument) (*CXArgument, error) {
	if arg1.Typ.Name != "i32" || arg2.Typ.Name != "i32" {
		return nil, errors.New("divI32: wrong argument type")
	}
	
	var num1 int32
	var num2 int32
	encoder.DeserializeAtomic(*arg1.Value, &num1)
	encoder.DeserializeAtomic(*arg2.Value, &num2)

	if num2 == int32(0) {
		return nil, errors.New("divI32: Division by 0")
	}

	output := encoder.SerializeAtomic(num1 / num2)

	return &CXArgument{Value: &output, Typ: MakeType("i32")}, nil
}
\end{lstlisting}

\section{Getters}
\label{implementation-getters}

\begin{lstlisting}[float=*, caption=Example of a Getter]
func (cxt *CXProgram) GetDefinition (name string) (*CXDefinition, error) {
	if mod, err := cxt.GetCurrentModule(); err == nil {
		var found *CXDefinition
		for _, def := range mod.Definitions {
			if def.Name == name {
				found = def
				break
			}
		}
		
		if found == nil {
			return nil, errors.New(fmt.Sprintf("Definition '\%s' not found", name))
		} else {
			return found, nil
		}
	} else {
		return nil, err
	}
}
\end{lstlisting}

\section{Makers}
\label{implementation-makers}

\begin{lstlisting}[float=*, caption=Example of a Maker]
func MakeModule (name string) *CXModule {
	return &CXModule{
		Name: name,
		Definitions: make([]*CXDefinition, 0),
		Imports: make([]*CXModule, 0),
		Functions: make([]*CXFunction, 0),
		Structs: make([]*CXStruct, 0),
	}
}
\end{lstlisting}

\section{Removers}
\label{implementation-removers}

\begin{lstlisting}[float=*, caption=Example of a Remover]
func (mod *CXModule) RemoveFunction (fnName string) {
	lenFns := len(mod.Functions)
	for i, fn := range mod.Functions {
		if fn.Name == fnName {
			if i == lenFns - 1 {
				mod.Functions = mod.Functions[:len(mod.Functions) - 1]
			} else {
				mod.Functions = append(mod.Functions[:i], mod.Functions[i+1:]...)
			}
			break
		}
	}
}
\end{lstlisting}

\begin{lstlisting}[float=*, caption=Example of a Remover]
func (expr *CXExpression) RemoveArgument () {
	if len(expr.Arguments) > 0 {
		expr.Arguments = expr.Arguments[:len(expr.Arguments) - 1]
	}
}
\end{lstlisting}

\section{Selectors}
\label{implementation-selectors}

\begin{lstlisting}[float=*, caption=Example of a Getter]
func (mod *CXModule) SelectFunction (name string) (*CXFunction, error) {
	var found *CXFunction
	for _, fn := range mod.Functions {
		if fn.Name == name {
			mod.CurrentFunction = fn
			found = fn
		}
	}

	if found == nil {
		return nil, errors.New("Desired function does not exist")
	}

	return found, nil
}
\end{lstlisting}

\section{Serialization}
\label{implementation-serialization}

%\begin{lstlisting}[float=*, caption=Example of a Getter]

%\end{lstlisting}

\section{Structs}
\label{implementation-structs}

\begin{lstlisting}[float=*, caption=Example of a Struct]
type CXProgram struct {
	Modules []*CXModule
	CurrentModule *CXModule
	CallStack *CXCallStack
	Outputs []*CXDefinition
	Steps []*CXCallStack
	ProgramSteps []*CXProgramStep
	Heap *[]byte
}
\end{lstlisting}
